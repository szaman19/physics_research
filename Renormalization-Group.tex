\documentclass{article}
\usepackage[utf8]{inputenc}
\usepackage{amsmath}
\usepackage{amsfonts}
\usepackage{amssymb}
\usepackage{amsthm}
\usepackage{epsfig}
\usepackage{epstopdf}
\usepackage{titling}
\usepackage{url}
\usepackage{array}
\usepackage{enumerate}
\usepackage[hmargin=2cm,vmargin=2.5cm]{geometry}
\title{Renormalization Group Notes}
\author{Shehtab Zaman }
\date{November 2017}

\begin{document}
\maketitle

\section{Background}

a group in quantum mechanics is any transformation
that leaves the physical system invariant.

In quantum electrodynamics, when computing physical quantity such
as scattering rate between eletrons, a power series in the coupling constant $\alpha $
are given by integrals over all possible momenta $ k$, which diverge.

We wish to consider the application of Renormalization Group methods to interacting
non-relativistic fermions especially weyl fermions.

In condensed matter physics, it is still possible to use RG a even with given
natural cutoffs (such as the inverse lattice spacing) and absense of ultraviolet infinities.

\section{Use Cases}

\subsection{Cubic Lattice(Shankar)}

Considering a d dimensional cubic lattice, with a real scalar field $ \phi(\hat{n})$ at
each site. The partition function is given by,

\begin{equation}
  \mathbf{Z} = \int \prod_{\hat{n}} d\phi(\hat{n}) e^{S(\phi{\hat{n}})}
\end{equation}

where $ S$ is the action.

The two-point function, the average of the correlation between the variables at two sites, is
given by

\begin{align}
  G(\hat{n_1},\hat{n_2}) &= G(\hat{n_1} - \hat{n_2}) \\
                        &= \langle \phi(\hat{n_1})\phi(\hat{n_2}) \rangle \\
                        &= \frac{1}{\mathbf{Z}} \int \prod_{\hat{n}} d\phi(\hat{n}) \phi(\hat{n_1})\phi(\hat{n_2}) e^{S(\phi{\hat{n}})}
\end{align}

The Fourier transform of the lattice vector field is given by,

\begin{equation}
  \phi(\hat{k}) = \frac{1}{V} \sum_{\hat{n}}e^{i\hat{k} - \hat{n}} \phi{\hat{n}}
\end{equation}

So we can write the partition function from eq.1 as,

\begin{equation}
  \mathbf{Z} = \int \prod_{|\hat {k}| \leq \frac{\pi}{a}} d\phi(\hat{k})e^{S(\phi(\hat{k}))}
\end{equation}
When considering problems of large seperations where $ \hat{n}_1 - \hat{n}_2$ is large,
in momentum space with small $\hat{k}$

So by using a cutoff momentum $ \frac{\Lambda}{s}$, we can have

\begin{equation}
  \phi_{<} =  \phi(k) \text{(slow modes)}
\end{equation}

\begin{equation}
  \phi_{>} =  \phi(k) \text{(fast modes)}
\end{equation}

In order to perform the Renormalization Group technique, we must perform the integral
over the $\phi_{<} $ fields only. So the first step for RG is to obtain an effective action
,$S^{'}(\phi _{<}) $, such that $e^{S^{'}(\phi _{<}) } $ produces all the slow momentum
correlation function when integrated over the slow modes.

\section{Critical Phenomenon}
Macroscopic variables like specific heat, magnetic
susceptibility or correlation length either diverge or
approachzero as $T \rightarrow T_c $
\section{Definitions}

\textbf{Scalar Field:} $(\phi(\hat{n}))$ A mathematical function that assigns
a value to each point/site labeled by vector $\hat{n} $. \newline
\textbf{Reciprocal Space:} $kx = 2\pi $ \newline
\textbf{Weyl Fermion:} Masslesss spin half quasi-particle with definite chirality. \newline
\textbf{Majorana Fermion:} Neutral spin half quasi-particle which is it's own anti-particle. \newline
\textbf{Response Function:} The\newline
\section{References}
Majorana, E., 1937, Il Nuovo Cimento (1924-1942) 14(4), 171.
\end{document}
