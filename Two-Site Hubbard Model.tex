\documentclass{article}
\usepackage[utf8]{inputenc}
\usepackage{amsmath}
\usepackage{amsfonts}
\usepackage{amssymb}
\usepackage{braket}
\usepackage{amsthm}
\usepackage{epsfig}
\usepackage{epstopdf}
\usepackage{titling}
\usepackage{url}
\usepackage{array}
\usepackage{enumerate}
\usepackage{booktabs}
\usepackage[hmargin=3cm,vmargin=2.5cm]{geometry}
\title{Fermionic Hubbard Model for Two-Site System}
\author{Shehtab Zaman }
\date{May 2017}

\begin{document}
\maketitle
\section{Hubbard Model Hamiltonian for Fermions}

\begin{equation}
\mathcal{H} = -t\sum_{\sigma} \left(f_{1 \sigma} ^{\dagger}f_{2 \sigma}+f_{2 \sigma} ^{\dagger}f_{1 \sigma} \right)
+ U\left(\hat{n}_{1\uparrow} \hat{n}_{1\downarrow} + \hat{n}_{2\uparrow} \hat{n}_{2\downarrow} \right)
\end{equation}

Where for site $ i$ and spin $ \sigma$, the creation, annihalation and number
operators are $f_{i \sigma} ^{\dagger}$,$f_{i \sigma} $,
$ \hat{n}_{i\sigma}$  respectively.

The number operator is defined as

$$ \hat{n}_{i\sigma} = f_{i \sigma} ^{\dagger}f_{i \sigma}$$
\section{Fock Space}
\subsection{Notes on Fock Space}
\subsubsection{Second Quantizaion}
The many body wavefunctions for bosons and fermions are difficult to work
with. So we use the occupation-number representation of "second quantizatin: formalism".
Essentially for fermions we consider the occupation number for each state. In
this example a half-filled two site system is considered and so each state
can either be 0 or 1, or each site can only one fermion with the same state and spin.
% \subsubsection{Creation and Annihilation operators}
\subsubsection{Anti-Commutation Relations}
Unlike bosons, fermions obey the anti-commutation relations. The \textbf{anti-commutator}
$\{A,B\} $ between two operators $ A$ and $ B$ are defined as
\begin{equation}
  \{A,B\}\equiv AB + BA
\end{equation}
The fermion creation and annhilation operators satisfies the relations
\begin{align}
  \{c_{k},c_{k\prime}\} &= 0 \\
  \{c^{\dagger}_{k},c^{\dagger}_{k\prime}\} &= 0 \\
  \{c^{\dagger}_{k},c_{k\prime}\} &= \delta_{k,k\prime}
\end{align}
\subsection{Fock Space Basis}
The Hubbard model does not change the total number of electron in the system. Thus we can consider
a half-filled $(N = 2)$ two site system. According to the Pauli Exclusion Principle, for a half-filled two-site system we
have 6 basis states.
\begin{table}[h]
\begin{center}
  \caption{Fock States}
  \label{tab: Title}
  \begin{tabular} {| lcc |}
      \hline
      $ \ket{\phi_{i}}$ & State & Spin Diagram \\ \hline
      $ \ket{\phi_{1}}$ & $c^{\dagger}_{2\uparrow}c^{\dagger}_{1\uparrow}\ket{} $   &   $ \uparrow ,\uparrow\text{\ } $\\    \hline
      $ \ket{\phi_{2}}$ & $c^{\dagger}_{1\downarrow}c^{\dagger}_{1\uparrow}\ket{} $ &   $  \uparrow\downarrow,\bigcirc $\\   \hline
      $ \ket{\phi_{3}}$ & $c^{\dagger}_{1\downarrow}c^{\dagger}_{2\uparrow}\ket{} $ &   $ \downarrow ,\uparrow \text{\ }$\\  \hline
      $ \ket{\phi_{4}}$ & $c^{\dagger}_{2\downarrow}c^{\dagger}_{1\uparrow}\ket{} $ &   $ \uparrow ,\downarrow \text{\ }$\\  \hline
      $ \ket{\phi_{5}}$ & $c^{\dagger}_{2\downarrow}c^{\dagger}_{2\uparrow}\ket{} $ &   $ \bigcirc,\uparrow\downarrow $\\    \hline
      $ \ket{\phi_{6}}$ & $c^{\dagger}_{1\downarrow}c^{\dagger}_{2\downarrow}\ket{} $ & $ \downarrow,\downarrow\text{\ } $\\ \hline
      \bottomrule
  \end{tabular}

\end{center}
\end{table}
\section{Two-Site Hubbard Model Matrix}
The Hubbard Hamiltoninan can be decomposed into the two decoupled Potential and Hopping Hamiltoninans.
Equation (1) can be written as,
\begin{equation}
\mathcal{H} = \mathcal{H}_t + \mathcal{H}_U
\end{equation}
Where, the Hopping Hamiltonian can be written as,
\begin{equation}
  \mathcal{H}_t = -t\sum_{\sigma} \left(f_{1 \sigma} ^{\dagger}f_{2 \sigma}+f_{2 \sigma} ^{\dagger}f_{1 \sigma} \right)
\end{equation}
and the Potential Hamiltonian can be written as,
\begin{equation}
  \mathcal{H}_U = U\left(\hat{n}_{1\uparrow} \hat{n}_{1\downarrow} + \hat{n}_{2\uparrow} \hat{n}_{2\downarrow} \right)
\end{equation}
\subsection{Hopping Hamiltonian in Occupation Number Basis}
  The hopping terms in the terms lie on the off diagonal terms of the matrix since the electrons must hop from one site to
  another.

  The Hamiltonian acting on a particular basis, will produce a superposition of occupation basis.
  \subsubsection{Basis Calculation}
  We need not consider \( \phi_1 \) and 9\( \phi_6\)since the both sites have electrons with the same spin, and
  thus all 4 terms of the hopping terms have zero contribution.

  Let us consider the action of \( \mathcal{H}_t\) on \(\phi_2\).
  \begin{align}
    \mathcal{H}_t \ket{\phi_2} &=
    -t\sum_{\sigma} \left(
      f_{1 \sigma} ^{\dagger}f_{2 \sigma}+
        f_{2 \sigma} ^{\dagger}f_{1 \sigma} \right)
          (f^\dagger_{1\downarrow}f^\dagger_{1\uparrow}) \ket{0}\\
          &= -t\left(
            f_{1 \uparrow} ^{\dagger}f_{2 \uparrow}+
              f_{2 \uparrow} ^{\dagger}f_{1 \uparrow} +
                f_{1 \downarrow} ^{\dagger}f_{2 \downarrow}+
                  f_{2 \downarrow} ^{\dagger}f_{1 \downarrow} \right) \left(f^\dagger_{1\downarrow}f^\dagger_{1\uparrow} \right) \ket{0} \\
          &= -t\left(
          f_{1 \uparrow} ^{\dagger}f_{2 \uparrow}\left(f^\dagger_{1\downarrow}f^\dagger_{1\uparrow} \right)+
            f_{2 \uparrow} ^{\dagger}f_{1 \uparrow} \left(f^\dagger_{1\downarrow}f^\dagger_{1\uparrow} \right)+
              f_{1 \downarrow} ^{\dagger}f_{2 \downarrow} \left(f^\dagger_{1\downarrow}f^\dagger_{1\uparrow} \right)+
                f_{2 \downarrow} ^{\dagger}f_{1 \downarrow} \right) \left(f^\dagger_{1\downarrow}f^\dagger_{1\uparrow} \right) \ket{0} \\
          &= -t\left(
            % f_{1 \uparrow} ^{\dagger}f_{2 \uparrow}\left(f^\dagger_{1\downarrow}f^\dagger_{1\uparrow} \right)+
              f_{2 \uparrow} ^{\dagger}f_{1 \uparrow} \left(f^\dagger_{1\downarrow}f^\dagger_{1\uparrow} \right)+
                % f_{1 \downarrow} ^{\dagger}f_{2 \downarrow} \left(f^\dagger_{1\downarrow}f^\dagger_{1\uparrow} \right)+
                  f_{2 \downarrow} ^{\dagger}f_{1 \downarrow} \left(f^\dagger_{1\downarrow}f^\dagger_{1\uparrow} \right)\right) \ket{0} \\
          &= -t\left(
            % f_{1 \uparrow} ^{\dagger}f_{2 \uparrow}\left(f^\dagger_{1\downarrow}f^\dagger_{1\uparrow} \right)+
              -f_{2 \uparrow} ^{\dagger}f^\dagger_{1\downarrow}f_{1 \uparrow}f^\dagger_{1\uparrow}+
                % f_{1 \downarrow} ^{\dagger}f_{2 \downarrow} \left(f^\dagger_{1\downarrow}f^\dagger_{1\uparrow} \right)+
                  f_{2 \downarrow} ^{\dagger}f^\dagger_{1\uparrow}f_{1 \downarrow}  f^\dagger_{1\downarrow} \right)\ket{0} \\
          &=-t\left(
            % f_{1 \uparrow} ^{\dagger}f_{2 \uparrow}\left(f^\dagger_{1\downarrow}f^\dagger_{1\uparrow} \right)+
              -f_{2 \uparrow} ^{\dagger}f^\dagger_{1\downarrow}\left(1 - \hat{n}_{1\uparrow}\right)+
                % f_{1 \downarrow} ^{\dagger}f_{2 \downarrow} \left(f^\dagger_{1\downarrow}f^\dagger_{1\uparrow} \right)+
                  f_{2 \downarrow} ^{\dagger}f^\dagger_{1\uparrow}\left(1 - \hat{n}_{1\downarrow}\right) \right)\ket{0} \\
          &=-t\left(
            % f_{1 \uparrow} ^{\dagger}f_{2 \uparrow}\left(f^\dagger_{1\downarrow}f^\dagger_{1\uparrow} \right)+
              -f_{2 \uparrow} ^{\dagger}f^\dagger_{1\downarrow}+
                % f_{1 \downarrow} ^{\dagger}f_{2 \downarrow} \left(f^\dagger_{1\downarrow}f^\dagger_{1\uparrow} \right)+
                  f_{2 \downarrow} ^{\dagger}f^\dagger_{1\uparrow} \right)\ket{} \\
          &= -t \ket{\phi_3} + -t\ket{\phi_4}
  \end{align}
  On equation (10) we have only have two terms that have non-zero contributions since the other two does not allow
  hopping from site 1 to 2 and we have a simplified hamiltonian action in equation (11).

  Using the relation
  \( f_{2 \uparrow} ^{\dagger}f_{1 \uparrow}f^\dagger_{1\downarrow}f^\dagger_{1\uparrow} = -f_{2 \uparrow} ^{\dagger}f^\dagger_{1\downarrow}f_{1 \uparrow}f^\dagger_{1\uparrow}\)
  (-1 since odd number of indices were switched) and
  \(f_{2 \downarrow} ^{\dagger}f_{1 \downarrow}f^\dagger_{1\downarrow}f^\dagger_{1\uparrow} = f_{2 \downarrow} ^{\dagger}f^\dagger_{1\uparrow}f_{1 \downarrow}  f^\dagger_{1\downarrow}\)
  we obtain equation (13).

  Also using the relation \(c_{1\sigma}c_{1\sigma}^\dagger = 1 - c_{1\sigma}^\dagger c_{1\sigma} = 1 - \hat{n}_{1 \dagger} \),
  we obtain equation (14)

  Finaly since the number operator , \( \hat{n}_{1 \sigma}\), acting on the vacuum state, \( \ket{0}\), is zero, we can get
  equation (15) and switching indices again to conform to the fock basis convention listed above, we get equation(16).

  Similar calculations can be down with the other basis get the Hopping Hamiltonian written as,
  \begin{equation}
    \mathcal{H}_t =
    \begin{bmatrix}
      0 & 0 & 0 & 0 & 0 & 0   \\
      0 & 0 & -t & -t & 0 & 0 \\
      0 & -t & 0 & 0 & -t & 0 \\
      0 & -t & 0 & 0 & -t & 0 \\
      0 & 0 & -t & -t & 0 & 0 \\
      0 & 0 & 0 & 0 & 0 & 0
    \end{bmatrix}
  \end{equation}
\subsection{Potential Hamiltonian in Occupation Number Basis}
The Potential terms occupy the diagonal of the occupation number matrix. In the basis defined above we can see that
only \(\ket{\phi_2} \) and \(\ket{\phi_5}\) have non-zero contributions.

So the Potential Hamiltonian can be written as,

\begin{equation}
  \mathcal{H}_U =
  \begin{bmatrix}
    0 & 0 & 0 & 0 & 0 & 0 \\
    0 & U & 0 & 0 & 0 & 0 \\
    0 & 0 & 0 & 0 & 0 & 0 \\
    0 & 0 & 0 & 0 & 0 & 0 \\
    0 & 0 & 0 & 0 & U & 0 \\
    0 & 0 & 0 & 0 & 0 & 0
  \end{bmatrix}
\end{equation}

From equation(6) we can write the complete Hamiltonian as,
\begin{equation}
  \mathcal{H} =
  \begin{bmatrix}
    0 & 0 & 0 & 0 & 0 & 0   \\
    0 & U & -t & -t & 0 & 0 \\
    0 & -t & 0 & 0 & -t & 0 \\
    0 & -t & 0 & 0 & -t & 0 \\
    0 & 0 & -t & -t & U & 0 \\
    0 & 0 & 0 & 0 & 0 & 0
  \end{bmatrix}
\end{equation}
\section{Exact Solution}


\end{document}
