\documentclass{article}
\usepackage[utf8]{inputenc}
\usepackage{amsmath}
\usepackage{amsfonts}
\usepackage{amssymb}
\usepackage{braket}
\usepackage{amsthm}
\usepackage{epsfig}
\usepackage{epstopdf}
\usepackage{titling}
\usepackage{url}
\usepackage{array}
\usepackage{enumerate}
\usepackage{booktabs}
\usepackage[hmargin=2cm,vmargin=2.5cm]{geometry}
\title{Fermionic Hubbard Model for Two-Site System}
\author{Shehtab Zaman }
\date{May 2017}

\begin{document}
\maketitle
\section{Hubbard Model Hamiltonian for Fermion}

\begin{equation}
\mathcal{H} = -t\sum_{\sigma} \left(f_{1 \sigma} ^{\dagger}f_{2 \sigma}+f_{2 \sigma} ^{\dagger}f_{1 \sigma} \right)
+ U\left(\hat{n}_{1\uparrow} \hat{n}_{1\downarrow} + \hat{n}_{2\uparrow} \hat{n}_{2\downarrow} \right)
\end{equation}

Where for site $ i$ and spin $ \sigma$, the creation, annihalation and number
operators are $f_{i \sigma} ^{\dagger}$,$f_{i \sigma} $,
$ \hat{n}_{i\sigma}$  respectively.

The number operator is defined as

$$ \hat{n}_{i\sigma} = f_{i \sigma} ^{\dagger}f_{i \sigma}$$
\section{Fock Space}
\subsection{Notes on Fock Space}
\subsubsection{Second Quantizaion}
The many body wavefunctions for bosons and fermions are difficult to work
with. So we use the occupation-number representation of "second quantizatin: formalism".
Essentially for fermions we consider the occupation number for each state. In
this example a half-filled two site system is considered and so each state
can either be 0 or 1, or each site can only one fermion with the same state and spin.
\subsubsection{Creation and Annihilation operators}
\subsubsection{Anti-Commutation Relations}
Unlike bosons, fermions obey the anti-commutation relations. The \textbf{anti-commutator}
$\{A,B\} $ between two operators $ A$ and $ B$ are defined as
\begin{equation}
  \{A,B\}\equiv AB + BA
\end{equation}
The fermion creation and annhilation operators satisfies the relations
\begin{align}
  \{c_{k},c_{k\prime}\} &= 0 \\
  \{c^{\dagger}_{k},c^{\dagger}_{k\prime}\} &= 0 \\
  \{c^{\dagger}_{k},c_{k\prime}\} &= \delta_{k,k\prime}
\end{align}
\subsection{Fock Space Basis}
The Hubbard model does not change the total number of electron in the system. Thus we can consider
a half-filled $(N = 2)$ two site system. According to the Pauli Exclusion Principle, for a half-filled two-site system we
have 6 basis states.
\begin{center}
\begin{tabular} {lcc}
    \toprule
    \multicolumn{3}{c}{Fock States} \\
    \midrule
    $ \ket{\phi_{i}}$ & State & Spin Diagram \\\midrule
    $ \ket{\phi_{1}}$ & $c^{\dagger}_{2\uparrow}c^{\dagger}_{1\uparrow}\ket{} $   &   $ \uparrow ,\uparrow\text{\ } $\\ \midrule
    $ \ket{\phi_{2}}$ & $c^{\dagger}_{1\downarrow}c^{\dagger}_{1\uparrow}\ket{} $ &   $  \uparrow\downarrow,\bigcirc $\\ \midrule
    $ \ket{\phi_{3}}$ & $c^{\dagger}_{1\downarrow}c^{\dagger}_{2\uparrow}\ket{} $ &   $ \downarrow ,\uparrow \text{\ }$\\ \midrule
    $ \ket{\phi_{4}}$ & $c^{\dagger}_{2\downarrow}c^{\dagger}_{1\uparrow}\ket{} $ &   $ \uparrow ,\downarrow \text{\ }$\\ \midrule
    $ \ket{\phi_{5}}$ & $c^{\dagger}_{2\downarrow}c^{\dagger}_{2\uparrow}\ket{} $ &   $ \bigcirc,\uparrow\downarrow $\\ \midrule
    $ \ket{\phi_{6}}$ & $c^{\dagger}_{1\downarrow}c^{\dagger}_{2\downarrow}\ket{} $ & $ \downarrow,\downarrow\text{\ } $\\ \midrule
    \bottomrule
\end{tabular}
\end{center}
\section{Two-Site Hubbard Model Matrix}
\begin{equation}
  \mathcal{H} =
  \begin{bmatrix}
    0 & 0 & 0 & 0 & 0 & 0 \\
    0 & U & -t & -t & 0 & 0 \\
    0 & -t & 0 & 0 & -t & 0 \\
    0 & -t & 0 & 0 & -t & 0 \\
    0 & 0 & -t & -t & U & 0 \\
    0 & 0 & 0 & 0 & 0 & 0
  \end{bmatrix}
\end{equation}
\section{Exact Solution}


\end{document}
