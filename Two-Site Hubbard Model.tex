\documentclass{article}
\usepackage[utf8]{inputenc}
\usepackage{amsmath}
\usepackage{amsfonts}
\usepackage{amssymb}
\usepackage{amsthm}
\usepackage{epsfig}
\usepackage{epstopdf}
\usepackage{titling}
\usepackage{url}
\usepackage{array}
\usepackage{enumerate}
\usepackage[hmargin=2cm,vmargin=2.5cm]{geometry}
\title{Fermionic Hubbard Model for Two-Site System}
\author{Shehtab Zaman }
\date{May 2017}

\begin{document}
\maketitle
\section{Hubbard Model Hamiltonian for Fermion}

\begin{equation}
\mathcal{H} = -t\sum_{\sigma} \left(f_{1 \sigma} ^{\dagger}f_{2 \sigma}+f_{2 \sigma} ^{\dagger}f_{1 \sigma} \right)
+ U\left(\hat{n}_{1\uparrow} \hat{n}_{1\downarrow} + \hat{n}_{2\uparrow} \hat{n}_{2\downarrow} \right)
\end{equation}

Where for site $ i$ and spin $ \sigma$, the creation, annihalation and number
operators are $f_{i \sigma} ^{\dagger}$,$f_{i \sigma} $,
$ \hat{n}_{i\sigma}$  respectively.

The number operator is defined as

$$ \hat{n}_{i\sigma} = f_{i \sigma} ^{\dagger}f_{i \sigma}$$
\section{Fock Space}
\subsection{Notes on Fock Space}

\subsection{Fock Space Basis}
The Hubbard model does not change the total number of electron in the system. Thus we can consider
a half-filled $(N = 2)$ two site system. According to the Pauli Exclusion Principle, for a half-filled two-site system we
have 6 basis states.
\begin{center}

\end{center}
\section{Two-Site Hubbard Model Matrix}
\begin{equation}
  \mathcal{H} =
  \begin{bmatrix}
    0 & 0 & 0 & 0 & 0 & 0 \\
    0 & U & -t & -t & 0 & 0 \\
    0 & -t & 0 & 0 & -t & 0 \\
    0 & -t & 0 & 0 & -t & 0 \\
    0 & 0 & -t & -t & U & 0 \\
    0 & 0 & 0 & 0 & 0 & 0
  \end{bmatrix}
\end{equation}
\section{Exact Solution}

\end{document}
