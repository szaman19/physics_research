\documentclass{article}
\usepackage[utf8]{inputenc}
\usepackage{amsmath}
\usepackage{amsfonts}
\usepackage{amssymb}
\usepackage{amsthm}
\usepackage{epsfig}
\usepackage{epstopdf}
\usepackage{titling}
\usepackage{url}
\usepackage{array}
\usepackage{enumerate}
\usepackage{ physics }
\usepackage[hmargin=3.5cm,vmargin=2.5cm]{geometry}
\title{Two Body Motion}
\author{Shehtab Zaman }
\date{May 2017}

\begin{document}
\maketitle
\section{Background}
Mehcanics of two particle that experience a mtual central conservative interaction.
\section{Lagrangian of Two-Body Problem}
\begin{equation}
  \mathcal{L} = \frac{1}{2}m_1 \dot{r_1}^2 + \frac{1}{2}m_2 \dot{r_2}^2
  -U\left(|\vec{r_1} - \vec{r_2}|\right)
\end{equation}
Lagrangian is has translation and rotational  inavirance.

Rotational invariance is due to the potential only haveing magnitude of the
vectors and not direction.

\section{Solving Lagrangian}
To decouple the lagrangian and the EL equations, change lagrangian to cener of
mass and relative mass frame. Translational invariance allows for the
transformation.


For center of Mass, radius $\vec{R}$ and relative mass, radius $ \vec{r}$, we
have

\begin{align}
\vec{R} &= \frac{m_1 \vec{r_1 } + m_2 \vec{r_2}}{m_1 + m_2}
 &
\vec{r} &= \vec{r_1}-\vec{r_2}
\end{align}

Using $\mu$ as the effective mass of

\begin{align}
  \mu &= \frac{m_1m_2}{m_1+ m_2} &
  M &= m_1 + m_2
\end{align}

We can decompose the into frames,

\begin{equation}
\mathcal{L} = \mathcal{L}_{CM} + \mathcal{L}_{Rel}
\end{equation}

Uing equations 2,3 to rewrite 1 we have

\begin{equation}
\mathcal{L} = \mathcal{L}_{CM} + \mathcal{L}_{Rel}
= \frac{1}{2}M\dot{R}^2 + \frac{1}{2}\mu(\dot{r})^2 - U(|\vec{r}|)
\end{equation}

Where we have,

\begin{align}
\mathcal{L}_{CM} &= \frac{1}{2}M\dot{R}^2 & \mathcal{L}_{Rel} = \frac{1}{2}\mu\dot{r}^2-U(|\vec{r}|)
\end{align}
\end{document}
