\documentclass{article}
\usepackage[utf8]{inputenc}
\usepackage{amsmath}
\usepackage{amsfonts}
\usepackage{amssymb}
\usepackage{amsthm}
\usepackage{epsfig}
\usepackage{epstopdf}
\usepackage{titling}
\usepackage{url}
\usepackage{array}
\usepackage{enumerate}
\usepackage{ physics }
\usepackage[hmargin=3.5cm,vmargin=2.5cm]{geometry}
\title{Phase Transitions Notes}
\author{Shehtab Zaman}
\date{Decembter 2017}

\begin{document}
\maketitle
\section{Phase Transitions}
\subsection{Latent Heat}
$$ C_x = T \left (\frac{\partial S}{ \partial T}\right)_x $$

To change from phase 1 to phase 2 at a constant temperature $T_c$, you
need to latent heat L.

$$ L = \Delta Q_{rev} = T_c(S_2- S_1) $$

where $S_1$ is the entropy of phase 1 and $S_2$ is the entropt of phase 2.
\newline
Consider the entropy discontinuity at a vapour-liquid transition. The number
of microstates $\Omega $ fora single gas molecule is prportional to its volume.
So we can write,

$$ \frac{\Omega_{vapour}}{\Omega_{liquid}} =
\left( \frac{V_{vapour}}{V_{liquid}}\right) ^ {N_A} $$

Considering that the density of vapour is roughly $10^3$ times smaller than
the density of the vapour, we can roughly see,

$$ L \approx 10 RRT_b$$

\subsubsection{Chemical Porential and Phase Changes}

"Gibbs functions is the quantity that must minimized when systems
are held at constant pressure and temperature"




\end{document}
